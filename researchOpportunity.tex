\documentclass[10pt,letterpaper,oneside]{article}

\input{./latexGoodPractices/preamble}

%----------------------------------------
% FILL THIS SECTION

\newcommand{\projectTitle}{Use a Descriptive Title}

% Internship Project, Master's Project, Doctoral Project
\newcommand{\projectLevel}{Master's Project}

\author{Fran\c{c}ois Pomerleau \\
       Laval University\\
       1065, av. de la Médecine \\
       Quebec, Qc \\
       Canada G1V 0A6 \\
       \texttt{<francois.pomerleau@ift.ulaval.ca>}
%       \and
%       Somebody Else \\
%       Laval University\\
%       1065, av. de la Médecine \\
%       Quebec, Qc \\
%       Canada G1V 0A6 \\
%       \texttt{<Somebody.else@ulaval.ca>}
}

% Change to your specific file
\addbibresource{./latexGoodPractices/exampleReferences.bib}

%----------------------------------------
% Page style

% Set the page size
\addtolength{\hoffset}{-1.0in} \addtolength{\voffset}{-0.75in}
\setlength{\textwidth}{7in} \setlength{\textheight}{8.25in}
\setlength{\headheight}{0.6in}
\setlength{\headsep}{0.2in}

\setlength{\footskip}{40pt}
\setlength{\fboxsep}{12pt}

% Set the paragraph skip
\setlength{\parskip}{3pt}

% Access to a counter for the number of pages
\usepackage{lastpage}

%----------------------------------------
% Title style

\newcommand{\makeCustomTitle}
{
\begin{center}
\large{---~\projectLevel{}~---}
\\
\vspace{-5pt}
\LARGE{\textbf{\projectTitle{}}}
\\
\vspace{5pt}
\small{Supervised by Prof.~F.~Pomerleau \texttt{<francois.pomerleau@ift.ulaval.ca>}}
\end{center}
}

%----------------------------------------
% Section style
\usepackage{sectsty}

% Set the section labeling font
\allsectionsfont{\textsf\bfseries}

%----------------------------------------
% Caption style
\usepackage[font=small, labelfont=bf, skip=5pt]{caption}

%----------------------------------------
% header style
\usepackage{fancyhdr}

% Set the page style for the document
\pagestyle{fancy}

% Define the basic page style

\fancyhf{}%

\fancyhead[L]{\includegraphics[height=0.45in]{UL_N}}
\fancyhead[C]{\raisebox{0.2in}{\textsc{Research Opportunity}}}
\fancyhead[R]{\includegraphics[height=0.45in]{norlab_logo_dark}}
\fancyfoot[L]{Last modification: \today}
\fancyfoot[R]{\thepage/\pageref*{LastPage}}

\renewcommand{\headrulewidth}{0.2pt}
\renewcommand{\footrulewidth}{0.2pt}


%----------------------------------------
% footnote style
\usepackage{fnpos}
% Fix the footnotes location
\makeFNbottom \makeFNbelow

% ---------------------------------------------------------------

% PDF setup
\hypersetup{%
    pdftitle={\projectLevel: \projectTitle},
    pdfauthor={\@author},
    pdfkeywords={research, project, robotics, norlab, Northern Robotics Lab},
    pdfsubject={},
    pdfstartview={},
    urlcolor=cyan,
    linkcolor=red,
}%

% ---------------------------------------------------------------

% Wrap text around a figure
\usepackage{wrapfig}
\setlength\intextsep{0pt}
% Fill the template with text
\usepackage{lipsum}

\begin{document}
\makeCustomTitle

% ---------------------------------------------------------------
\section*{Project Proposal}

\begin{wrapfigure}{R}{0.35\textwidth}
\centering
\includegraphics[draft, width=0.3\textwidth]{./figs/overview.pdf}
\caption{
Replace the file \texttt{./figs/overview.pdf} with a photo or diagram that catch the eye.
}
\label{fig:overview}
\end{wrapfigure}

Limit the number of references to two \cite{Pomerleau2013,Pomerleau2014}.
\lipsum[1-2]


% ---------------------------------------------------------------
\section*{Research Environment}

The project will be hosted by the Northern Robotics Laboratory (norlab) located on the main campus of Laval University.
The university was established in \num{1663}, making it the oldest academic institution in Canada and the first school to offer higher education in French.
It currently enrolls \num{50000} students, from which around \num{9000} are at the postgraduate level.
Norlab is specialized in mobile and autonomous systems working in winter or difficult conditions. 
We aim at investigating new challenges related to navigation algorithms to push the boundary of what is currently possible to achieve with a mobile robot in real-life conditions. 
The current focus of the laboratory is on localization algorithms designed for laser sensors (lidar) and 3D reconstruction of the environment.







% ---------------------------------------------------------------
\printbibliography


\end{document}
